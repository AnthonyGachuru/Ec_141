%% LyX 2.3.2 created this file.  For more info, see http://www.lyx.org/.
%% Do not edit unless you really know what you are doing.
\documentclass[english]{article}
\usepackage[T1]{fontenc}
\usepackage[latin9]{inputenc}
\usepackage{geometry}
\geometry{verbose,tmargin=1in,bmargin=1in,lmargin=1in,rmargin=1in}
\setlength{\parskip}{\smallskipamount}
\setlength{\parindent}{0pt}
\usepackage{amsmath}
\usepackage{amsthm}
\usepackage{amssymb}
\usepackage{setspace}
\PassOptionsToPackage{normalem}{ulem}
\usepackage{ulem}
\onehalfspacing
\usepackage{babel}
\begin{document}
\uline{Ec141, Spring 2020 }

\emph{Midterm 1}

\textbf{Please read each question carefully.} Start each question
on a new bluebook page. The use of calculators and other computational
aides is not allowed. Good luck!

\medskip{}

{[}1{]}\quad{}\textbf{{[}5 Points{]}}\quad{}Please write your full
name and student ID on the first page of your pdf file of exam solutions.\medskip{}

{[}2{]}\quad{}\textbf{{[}15 Points{]}}\quad{}The Undergraduate Dean
has been collecting data on the high school GPA ($X$) of incoming
students for a long, long time. She has also kept track of 1st semester
GPA ($Y$) for enrolled students over the same period of time. She
would like to be able to predict a student's 1st semester GPA using
their high school GPA. She reports to you the following means, variances
and a covariance for $X$ and $Y$: 
\[
\mu_{X}=\frac{12}{5},\mu_{Y}=2
\]
and
\[
\sigma_{X}^{2}=1/6,\sigma_{Y}^{2}=1/4,\sigma_{XY}=1/5.
\]
Because she has collected such a large sample you are free to treat
these numbers as if they were population quantities.

\qquad{}{[}a{]}\quad{}\textbf{{[}5 Points{]}}\quad{}Calculate the
$\alpha$ and $\beta$ associated with the (mean square error minimizing)
linear predictor of $Y$ given $X$, $\mathbb{E}^{*}[Y|X]=\alpha+\beta X$?

\qquad{}{[}b{]}\quad{}\textbf{{[}10 Points{]}}\quad{}Say $\sigma_{X}^{2}=0$
(e.g., only students with 4.0 GPAs have ever been admitted). What
is the best linear predictor of $Y$ given $X$ in this case (you
may assume all other parameters stay the same)? Why? \textbf{{[}2-4
sentences{]}}.

\medskip{}

{[}3{]}\quad{}\textbf{{[}25 Points{]}}\quad{}Consider the following
statistical model for the logarithm of daily city-wide sales of Bob
Dylan's landmark \emph{Christmas in the Heart} album:
\[
\ln S=\alpha_{0}+\beta_{0}R+\gamma_{0}P+U,\thinspace\thinspace\mathbb{E}\left[\left.U\right|R,P\right]=0,
\]
where $R$ is the number of times a song from the album is played
on KALX on the given day, and $P$ is the price of the album (which
varies across your sample due to various (exogenous) record label
promotions, holiday sales and so on). A friend estimates $\theta_{0}=\left(\alpha_{0},\beta_{0},\gamma_{0}\right)'$
by the method of least squares. She claims that $\sqrt{N}\left(\hat{\theta}-\theta_{0}\right)\overset{D}{\rightarrow}\mathcal{N}\left(0,\Lambda_{0}\right)$
and reports the following:
\[
\hat{\theta}=\left(\begin{array}{c}
1.0\\
0.01\\
-0.51
\end{array}\right),\thinspace\thinspace\frac{\hat{\Lambda}}{N}=\left(\begin{array}{ccc}
0.25 & -0.002 & 0.010\\
-0.002 & 0.01 & 0.005\\
0.010 & 0.005 & 0.03
\end{array}\right).
\]
\quad{}{[}a{]}\quad{}\textbf{{[}2 Points{]}}\quad{}Calculate a
95 confidence interval for $\beta_{0}$.

\quad{}{[}b{]}\quad{}\textbf{{[}5 Points{]}}\quad{}Your friend
would like to test the hypothesis that ``for Bob Dylan one song on
the radio is as good as cutting record price by \$1'' (a phrase used
by her record store boss). Explain why this corresponds to:
\begin{align*}
H_{0} & :\beta_{0}=-\gamma_{0}\\
H_{1} & :\beta_{0}\neq-\gamma_{0}
\end{align*}
\quad{}{[}c{]}\quad{}\textbf{{[}5 Points{]}}\quad{}We can re-write
$H_{0}$ as
\[
H_{0}:C\theta=c
\]
Provide the appropriate forms for $C$ and $c$.

\quad{}{[}d{]}\quad{}\textbf{{[}5 Points{]}}\quad{}How many restrictions
on $\theta$ does $H_{0}$ imposes?

\quad{}{[}e{]}\quad{}\textbf{{[}5 Points{]}}\quad{}Calculate the
Wald statistics for $H_{0}$. Can we reject with size $\alpha=0.05$?

\quad{}{[}f{]}\quad{}\textbf{{[}8 Points{]}}\quad{}Now formalize
and test the hypothesis that ``for Bob Dylan one song on the radio
is as good as cutting record price by \$3''.

\medskip{}

{[}4{]}\quad{}\textbf{{[}30 Points{]}}\quad{}Consider the following
statistical model for the earnings of Berkeley students
\[
Y=\alpha+\beta G+\gamma A+U,\thinspace\mathbb{E}[U|G,A]=0,
\]
where $G$ equals one if the student graduated and zero if they dropped
out and $A$ equals one if at least one of the student\textquoteright s
parents graduated from college and zero otherwise.

\qquad{}{[}a{]}\quad{}\textbf{{[}5 Points{]}}\quad{}You read in
the Oakland Tribune newspaper that Berkeley graduates earn an average
of \$75,000 per year nationwide, while the earnings of dropouts average
only \$15,000. Express this population earnings difference between
Berkeley graduates and dropouts in terms of the statistical model
given above.

\qquad{}{[}b{]}\quad{}\textbf{{[}5 Points{]}}\quad{}Under what
conditions is it true that $\beta=\ensuremath{\$60,000}$? Do you
think these conditions are likely to be true in practice? Briefly
explain your answer \textbf{{[}3-5 sentences{]}}.

\qquad{}{[}c{]}\quad{}\textbf{{[}5 Points{]}}\quad{}The same article
reports that among Berkeley graduates, three fourths come from families
where at least one parent completed college, while among all former
students (i.e., graduates and dropouts) only seven twelfths come from
such families. It also states that the overall (i.e., unconditional)
graduation rate at Berkeley is two-thirds. Among dropouts, what fraction
come from families where at least one parent completed college?

\qquad{}{[}d{]}\quad{}\textbf{{[}5 Points{]}}\quad{}Assume $\gamma=\ensuremath{\$25,000}$.
Using your answers in parts (a) and (c) solve for $\beta$. What is
the expected earnings gain associated with graduating from Berkeley
holding parent\textquoteright s education (i.e., A) constant? Briefly
comment on why your answer differs from the earnings gap between graduates
and dropouts reported by the Tribune \textbf{{[}3-5 sentences{]}}.

\qquad{}{[}e{]}\quad{}\textbf{{[}5 Points{]}}\quad{}You are considering
dropping out of Cal to spend more time on Telegraph Avenue. What is
the (approximate) expected earnings loss associated with this decision?
Explain \textbf{{[}3-5 sentences{]}}.

\qquad{}{[}f{]}\quad{}\textbf{{[}5 Points{]}}\quad{}You move to
Oakland upon graduation, your neighbor to the left tells you that
he dropped out of Berkeley during the Free Speech Movement, your neighbor
to the right tells you that he graduated from Berkeley about the same
time. What is your expectation of the annual earnings of your two
neighbors? Explain \textbf{{[}3-5 sentences{]}}.

\medskip{}

{[}5{]}\quad{}\textbf{{[}20 Points{]}}\quad{}The World Health Organization
has contracted you to design a randomized experiment evaluating the
efficacy of zinc supplements on diarrhea prevalence (measured as the
number of episodes in the one hundred days prior to surveying). Let
$Y\left(1\right)$ be the potential number of episodes of diarrhea
if taking zinc supplements and $Y\left(0\right)$ the control potential
outcome. A baseline survey of your target population yields a diarrhea
prevalence of 10 days per one hundred days with a standard deviation
of 5 days. Let $N$ be your target sample size and assume that half
of respondents will be randomly assigned to treatment. Assume that
the variance of $Y\left(1\right)$ and $Y\left(0\right)$ are equal
to each other. Also assume that no respondents in your baseline survey
were taking zinc supplements.

\qquad{}{[}a{]}\quad{}\textbf{{[}10 Points{]}}\quad{}Derive an
expression for the ex ante probability ($\beta$) that you reject
the null of no effect in favor of a \emph{one-sided} alternative of
a negative effect (i.e., treatment reduces diarrhea). Let $\alpha$
denote the size of your test and $\theta$ the ATE. Carefully explain
your reasoning and notation \textbf{{[}4-6 sentences{]}}.

\qquad{}{[}b{]}\quad{}\textbf{{[}5 Points{]}}\quad{}Assume that
$\theta=-10$. How large would $N$ need to be to ensure an ex ante
rejection probability of 95 percent (for a test with size $\alpha=0.05$).

\qquad{}{[}c{]}\quad{}\textbf{{[}5 Points{]}}\quad{}You ultimately
design an experiment with power of $\beta=0.90$ and size $\alpha=0.05.$
In the end you find no effect of zinc supplements on the prevalence
of diarrhea (i.e., you fail to reject the null of no effect). Prior
to the experiment you believed that the probability that zinc supplements
reduced the prevalence of diarrhea was 0.75. What is your belief after
your null finding? Explain \textbf{{[}3-5 sentences{]}}.
\end{document}
